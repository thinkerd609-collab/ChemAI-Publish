2. Fundamentals of Raman and FTIR Spectroscopy for Microplastic Analysis
2.1 Principles of Raman Spectroscopy
Raman spectroscopy is grounded in the phenomenon of inelastic light scattering, first described quantum mechanically by C. V. Raman in 1928 and now constituting one of the most information-rich analytical tools in polymer characterization. When monochromatic laser radiation of frequency ν₀ irradiates a sample, the overwhelming majority of photons are scattered elastically (Rayleigh scattering) at the incident frequency; however, a small fraction—on the order of 10⁻⁸ of incident photons—undergoes inelastic scattering resulting in a frequency shift proportional to the vibrational energy levels of the molecule. Photons scattered at lower frequency (ν₀ − ν_vib) give rise to the Stokes lines, which are conventionally reported as positive Raman shifts, while Anti-Stokes scattering at higher frequency (ν₀ + ν_vib) reflects molecules already in excited vibrational states and has significantly lower intensity at ambient temperature, consistent with Boltzmann population statistics (Sunil et al., 2025).
The fundamental selection rule for Raman activity requires a change in the molecular polarizability (α) during the vibration, expressed formally as:
(∂α/∂Q)₀ ≠ 0
where Q denotes the normal coordinate of the vibration. This polarizability criterion is mechanistically complementary to the IR selection rule, a distinction of substantial practical importance for microplastic analysis. Raman spectra of synthetic polymers are typically acquired over the 400–3500 cm⁻¹ range; the diagnostic fingerprint region (400–1800 cm⁻¹) contains C–C, C–C skeletal, and ring deformation modes, while the 2800–3200 cm⁻¹ region captures C–H stretching vibrations (Sunil et al., 2025; Zhu et al., 2025). Among common environmental polymers, polyethylene (PE) exhibits diagnostic Raman bands at approximately 1060, 1130, 2848, and 2881 cm⁻¹; polypropylene (PP) is identified via bands at 808, 973, and 2951 cm⁻¹; and polystyrene (PS) displays a highly characteristic aromatic ring breathing mode at 1001 cm⁻¹ alongside C–H stretching near 3054 cm⁻¹, which collectively allow unambiguous polymer-type assignment even on complex environmental matrices (Table 2).
The spatial resolution achievable with confocal Raman microspectroscopy, typically ~1 μm with a 532 nm laser, renders this technique uniquely capable of characterizing sub-10-μm particles that fall below the detection threshold of conventional FTIR instrumentation. This is especially critical for nanoplastic detection and for distinguishing individual particles within heterogeneous environmental samples. Furthermore, because water is a weak Raman scatterer—exhibiting only a broad envelope near 3400 cm⁻¹ that does not overlap with most polymer bands—Raman spectroscopy is intrinsically suited to the analysis of aqueous suspensions, wet sediment extracts, and biota-associated microplastics without prior desiccation (Müller et al., 2025).
2.2 Principles of Fourier-Transform Infrared (FTIR) Spectroscopy
Fourier-transform infrared spectroscopy operates on the basis of molecular absorption of mid-infrared radiation (400–4000 cm⁻¹), whereby photons are absorbed when their energy matches discrete vibrational transition energies within a molecule. The selection rule governing IR activity stipulates that a vibrational mode must produce a change in the molecular dipole moment (μ) to be IR-active:
(∂μ/∂Q)₀ ≠ 0
This complementarity with the Raman polarizability rule is particularly consequential for polymers possessing a center of symmetry, where the mutual exclusion principle applies; however, for the low-symmetry synthetic polymers encountered as microplastics in environmental matrices, partial overlap between IR-active and Raman-active modes is common, and joint interpretation of both spectra yields a more complete vibrational fingerprint (Cabot et al., 2024).
In practice, FTIR spectrometers employ a Michelson interferometer to encode all wavelengths simultaneously, and the resulting interferogram is converted by Fourier transformation into a conventional absorbance spectrum. Two principal sampling modalities are employed in microplastic analysis: attenuated total reflectance (ATR)-FTIR, in which the evanescent IR wave penetrates 1–2 μm into the sample surface upon total internal reflection at the crystal interface, and transmission-mode micro-FTIR, used with thin polymer films or particles deposited on IR-transparent substrates such as calcium fluoride or zinc selenide windows (Smolen et al., 2025). Characteristic FTIR absorptions include the strong C–H stretching doublet of PE at 2916 and 2849 cm⁻¹, the methyl rocking band of PP at 1167 cm⁻¹, the C=O ester stretching of poly(ethylene terephthalate) (PET) at ~1715 cm⁻¹, and the C–Cl stretching modes of polyvinyl chloride (PVC) at 615–690 cm⁻¹ (Table 2).
The principal practical limitation of FTIR spectroscopy in the environmental context is its sensitivity to water, which absorbs strongly across the 1500–3800 cm⁻¹ region due to O–H stretching and H–O–H bending modes, thereby masking diagnostic polymer bands and rendering direct analysis of aqueous matrices impractical without prior drying. In addition, spatial resolution in focal-plane-array (FPA) FTIR imaging is constrained to approximately 10–20 μm in diffraction-limited configurations, meaning that particles below ~10 μm are systematically underrepresented in FTIR-based abundance surveys—an important source of bias in environmental monitoring studies (Cabot et al., 2024; Zhu et al., 2025). Notwithstanding these constraints, FTIR remains the preferred approach for high-throughput bulk and filter-based analysis, where large-area particle inventories are required.
2.3 Comparative Advantages and Complementary Nature of Raman and FTIR
The two spectroscopic techniques occupy non-overlapping but complementary analytical niches in microplastic research, as summarized in Table 1. Raman spectroscopy excels in the characterization of small (< 10 μm) particles—including fibres, film fragments, and nascent nanoplastics—in aqueous environments, and its confocal geometry enables true three-dimensional depth profiling of particle surfaces. FTIR, by contrast, offers superior throughput for large-area filter analysis (e.g., membrane filters used in water sampling), is less susceptible to fluorescent interference from biological co-extractants, and requires no laser alignment or fluorescence quenching pretreatment (Müller et al., 2025).
Table 1. Comparative overview of Raman and FTIR spectroscopy for microplastic analysis.
Parameter	Raman Spectroscopy	FTIR Spectroscopy
Physical basis	Inelastic (Raman) scattering	Absorption of IR photons
Spectral range	400–3500 cm⁻¹ (Stokes)	400–4000 cm⁻¹ (mid-IR)
Water interference	Minimal (water is Raman-inactive)	Severe (strong O–H absorption)
Spatial resolution	~1 μm (confocal laser)	~10–20 μm (ATR-FTIR)
Minimum particle size	< 1 μm (Raman mapping)	~10–20 μm
Fluorescence susceptibility	High (organic matter, sediment)	Low
Sample preparation	Minimal; aqueous compatible	KBr pellet, ATR crystal, or filter
Throughput	Slow (point/mapping mode)	Fast (bulk/filter imaging)
Portable cost range	~$2,000–5,000 (emerging)	~$500–2,000 (handheld)

A particularly insidious challenge for Raman analysis of environmental microplastics is fluorescence emission from co-occurring natural organic matter, humic substances, and biological films, which can overwhelm the weak Raman signal and produce broad featureless spectral backgrounds. Mitigation strategies include near-infrared (NIR) excitation (785 or 1064 nm) to reduce fluorescence quantum yield, photobleaching protocols, baseline correction algorithms, and shifted-excitation Raman difference spectroscopy (SERDS). Conversely, FTIR suffers from matrix spectral overlap and particle size-dependent scattering artefacts (Mie resonances) that distort band shapes and intensities, requiring preprocessing corrections that introduce additional uncertainty into quantitative analysis (Smolen et al., 2025). The complementary application of both techniques—using FTIR for initial bulk screening and Raman for confirmation of sub-10-μm particles—represents the current methodological gold standard for comprehensive microplastic characterization in environmental matrices, and this dual-modality data architecture also offers compelling advantages for machine-learning classification, as elaborated in Section 3.
2.4 Instrumentation: Benchtop versus Portable Systems
Benchtop Raman and FTIR spectrometers remain the analytical reference standard, offering superior spectral resolution (< 1 cm⁻¹), extended wavenumber coverage, high signal-to-noise ratios, and coupling to optical microscopes for spatially resolved analysis. Laboratory-grade confocal Raman microscopes (e.g., WITec, Renishaw, Horiba) operate at spatial resolutions approaching the diffraction limit (~0.5–1 μm) and are capable of automated mapping of entire filter surfaces at sub-10-μm resolution, generating particle count, size distribution, polymer type, and morphological data in a single analytical workflow (Zhu et al., 2025). FPA-FTIR imaging systems similarly enable large-format simultaneous spectral mapping, with detector arrays of up to 64 × 64 elements reducing acquisition times for full-filter analyses from hours to minutes.
The emergence of miniaturized, handheld, and portable spectrometers represents a transformative development for environmental monitoring in resource-constrained settings. Portable FTIR instruments, including ATR-based devices from manufacturers such as Agilent (4300 Handheld FTIR), Thermo Fisher (Nicolet Summit), and Ocean Insight, are now commercially available at price points between approximately $500 and $2,000, providing spectral coverage from ~650 to 4000 cm⁻¹ and sufficient resolution (4–8 cm⁻¹) for polymer-type identification of particles ≥20 μm (Müller et al., 2025; Smolen et al., 2025). Portable Raman devices operating at 785 nm excitation are available in the $2,000–5,000 range, though their performance for environmental microplastic detection remains more limited relative to benchtop systems, primarily due to reduced laser power, detector sensitivity, and fixed sampling geometry.
The relevance of portable instrumentation to field campaigns in South Asia, particularly Pakistan, cannot be overstated. Pakistan ranks among the world’s most severely affected nations with respect to plastic pollution, with extensive riverine transport through the Indus Basin and high loads of microplastics in agricultural soils, irrigation water, and coastal sediments. Laboratory infrastructure in many monitoring sites is limited, cold-chain sample transport is logistically challenging, and analytical backlogs in centralized facilities routinely compromise sample integrity. Handheld FTIR spectrometers offer a practical pathway to in-situ, point-of-collection polymer identification that circumvents these constraints, supporting real-time decision making and large-scale geographic sampling that would be impractical with benchtop-only protocols (Cabot et al., 2024; Smolen et al., 2025). The integration of portable spectral data into machine-learning pipelines—trained on curated benchtop reference libraries—represents a promising strategy to bridge the analytical performance gap between field and laboratory instruments.
2.5 Challenges in Environmental Microplastic Analysis
Despite the power of Raman and FTIR spectroscopy for polymer identification, their application to environmental microplastics is complicated by a suite of matrix-specific challenges that substantially degrade spectral quality and classification accuracy. Photooxidative and hydrolytic weathering of plastic surfaces in the environment induces chemical modifications—including carbonyl group formation, chain scission, dehydrochlorination in PVC, and surface chalking—that alter diagnostic band positions and relative intensities, producing spectral signatures that deviate systematically from pristine polymer reference libraries (Zhu et al., 2025). This spectral drift is especially pronounced for aged PE and PP collected from UV-exposed surface environments, where the characteristic fingerprint band intensities are diminished and a broad carbonyl absorption envelope emerges near 1700–1750 cm⁻¹.
Biofouling represents an additional confounding factor: biofilm formation and adsorption of humic substances onto particle surfaces introduce new spectral bands and broad fluorescent backgrounds that partially or wholly obscure polymer signals. Pre-treatment protocols involving enzymatic digestion (proteinase K, cellulase), chemical oxidation (H₂O₂, KOH, Fenton reagent), and density separation partially mitigate biofouling but risk chemically modifying labile polymer surfaces or dissolving low-density microplastics, thereby introducing sample losses and spectral artefacts of their own (Müller et al., 2025). Spectral library coverage is a further limitation: most commercially available and open-access databases (e.g., KnowItAll, SIFSAS, Open Specy) have been compiled primarily from pristine technical-grade polymers, and contain limited entries for mixed-polymer composites, plasticizer-containing formulations, and environmentally weathered particles, which are the predominant forms encountered in real samples.
At the nanoscale (<1 μm), both techniques face fundamental sensitivity constraints. Conventional Raman spectroscopy requires extended acquisition times (minutes to hours per particle) at the nanoscale and is challenged by particle photodegradation under focused laser irradiation. Surface-enhanced Raman spectroscopy (SERS) using gold or silver nanoparticle substrates has been explored to amplify signals from nanoplastics, but substrate reproducibility, matrix effects, and calibration remain active areas of methodological development (Sunil et al., 2025). These analytical challenges collectively underscore the urgent need for automated, high-throughput spectral processing pipelines capable of operating robustly across heterogeneous, degraded, and noisy environmental spectra—a requirement that machine learning is particularly well-positioned to address.
Table 2. Characteristic Raman and FTIR spectral bands for common environmental microplastic polymers.
Polymer	Abbreviation	Key Raman Bands (cm⁻¹)	Key FTIR Bands (cm⁻¹)
Polyethylene	PE	1060, 1130, 2848, 2881	720, 1465, 2916, 2849
Polypropylene	PP	808, 973, 1153, 2951	840, 998, 1167, 2957
Polystyrene	PS	620, 1001, 1602, 3054	698, 758, 1452, 3026
Poly(ethylene terephthalate)	PET	1095, 1615, 1726, 3064	723, 1095, 1715, 3444
Polyvinyl chloride	PVC	638, 696, 1428, 2975	615, 690, 1250, 2970
Nylon-6	PA-6	1063, 1124, 1440, 2931	689, 1200, 1545, 3303

2.6 Transition to Machine-Learning Integration
The spectroscopic principles outlined above converge on a fundamental data-science challenge: the generation of large-dimensional spectral datasets—each comprising hundreds to thousands of wavenumber data points per particle—that must be classified rapidly, accurately, and consistently across variable sample matrices, instrument configurations, and weathering states. Manual spectral interpretation by expert analysts is impractical at the scales demanded by contemporary environmental monitoring programs, which may involve the characterization of tens of thousands of particles per study. Furthermore, inter-laboratory reproducibility studies have demonstrated that even experienced analysts achieve only moderate agreement on polymer classification from degraded or mixed spectra, with reported intra-laboratory concordance rates ranging from 60% to 90% depending on sample complexity (Cabot et al., 2024; Müller et al., 2025).
Machine-learning and deep-learning algorithms are inherently well-matched to the structural characteristics of vibrational spectroscopic data: the fixed wavenumber axis defines a high-dimensional but consistently structured feature space, spectral libraries provide labeled training data at sufficient scale for supervised learning, and the pattern-recognition strengths of convolutional neural networks (CNNs) and gradient-boosted ensemble methods map naturally onto the problem of identifying diagnostic band combinations in the presence of complex backgrounds and spectral distortions. Critically, the complementarity of Raman and FTIR data—each interrogating distinct but overlapping subsets of molecular vibrational modes—creates a multimodal fusion opportunity wherein combined spectral inputs may yield classification accuracies exceeding those achievable with either modality alone (Smolen et al., 2025; Zhu et al., 2025). The design of effective machine-learning pipelines must, however, account rigorously for the spectral preprocessing requirements, library representativeness constraints, and instrument-to-instrument variability discussed in this section, lest models trained on curated laboratory data fail to generalize to the messy realities of field-collected samples.
These fundamental spectroscopic principles provide the essential data foundation upon which modern machine-learning algorithms operate, as detailed in the following section.
Disclaimer
All statements in this section are based exclusively on the cited references.
