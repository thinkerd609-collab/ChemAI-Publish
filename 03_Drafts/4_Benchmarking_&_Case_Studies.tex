4. Benchmarking & Case Studies
4.1 Performance on Pristine vs. Weathered/Environmental Samples
A recurring and critically important observation in the microplastic spectroscopy literature is the systematic divergence between classification accuracy reported on pristine polymer reference libraries and that achieved on authentic environmental samples. Across the studies reviewed, accuracies of 95–99% are routinely attained when training and testing sets are drawn from the same curated laboratory collection of unweathered, additive-free polymer standards (Zhang et al., 2025; Li et al., 2026). By contrast, when the identical model architectures are applied to field-collected particles—whether extracted from river water, marine sediment, agricultural soil, or atmospheric deposition filters—accuracy commonly declines to the range of 80–92%, with F1-scores frequently falling below 0.90 (Zhu et al., 2025; Müller et al., 2025; Khanam et al., 2025). This accuracy gap is not an incidental artefact of poor model design but reflects fundamental physicochemical transformations undergone by plastics during environmental ageing.
Photooxidative degradation under UV irradiation introduces carbonyl functionalities (ketones, aldehydes, carboxylic acids) through Norrish-type I and II reactions, generating a broad absorption envelope between 1700 and 1750 cm⁻¹ in FTIR spectra that partially obscures or shifts diagnostic fingerprint bands for PE, PP, and PS. In Raman spectra, surface chalking and the accumulation of fluorescent chromophores derived from hydroperoxide decomposition elevate background noise and suppress the signal-to-noise ratio of characteristic bands, particularly at 532 nm excitation (Smolen et al., 2025). Hydrolytic degradation of PET and polycarbonate produces chain-end carboxylic acid and hydroxyl groups detectable by FTIR, while chain scission reduces particle size into the sub-10-μm range, simultaneously increasing the challenge of spectroscopic detection and rendering particle-number-based abundance estimates highly sensitive to analytical detection limits. Biofouling superimposes protein amide I and amide II bands (~1650 and ~1540 cm⁻¹), polysaccharide ring modes (~1030 cm⁻¹), and lipid C–H stretches on polymer spectra, creating composite signatures that neither pristine-library SVM nor CNN models handle reliably without domain-adapted retraining (Cabot et al., 2024).
Müller et al. (2025) conducted a systematic comparison in which a ResNet-1D model trained exclusively on pristine Raman spectra was evaluated against three test sets of increasing environmental complexity: pristine particles, laboratory-aged particles (UV-irradiated for 500 h), and field-collected sediment particles. Accuracy declined from 97% to 91% to 88%, respectively, with the principal error mode being the misclassification of carbonylated PE as PET due to the overlapping C=O band introduced by weathering. Retraining with a dataset augmented by spectra of UV-aged and biofouled particles partially recovered accuracy (to 93% on field samples), underscoring that domain-adaptive retraining is not an optional refinement but a prerequisite for deployment in environmental monitoring programs. The critical methodological implication is unambiguous: published accuracy figures derived solely from pristine laboratory datasets are not a reliable predictor of real-world analytical performance, and the field requires standardized weathered-particle test sets against which all new models must be evaluated.
4.2 Case Studies from Real Environmental Matrices (Water, Sediment, Soil, Air)
Table 5 compiles ten representative case studies spanning the principal environmental matrices in which microplastic monitoring programs are conducted, drawn from the surveyed literature. Across these studies, several cross-cutting patterns emerge that merit critical discussion beyond the individual accuracy figures.
In riverine and freshwater systems, the dominant polymers detected by ML-assisted Raman and FTIR classification are PE and PP—reflecting both their high global production volumes and their persistence in the environment—with polyester (PET) microfibers increasingly prominent in sampling locations proximal to urban or industrial textile sources (Zhu et al., 2025). The 1D-CNN applied to Raman spectra of river water extracts by Zhu et al. (2025) achieved 89% accuracy on field samples versus 96% on pristine library spectra from the same study, with the primary failure mode being misclassification of biofouled PP as polyamide due to protein band overlap with PP amide-region artefacts. High-throughput automated Raman mapping of membrane filters—enabled by motorized stage control and automated particle detection from brightfield images—allowed the analysis of up to 500 particles per sample in under two hours, a throughput improvement of approximately one order of magnitude over manual identification protocols.
In marine and coastal sediment matrices, 2D-CNN applied to Raman hyperspectral image stacks demonstrated notably higher accuracy (91%) than single-spectrum 1D approaches on the same particle population, attributable to the incorporation of morphological features—aspect ratio, surface texture heterogeneity, particle boundary regularity—that complement purely spectral classification (Zhu et al., 2025). Marine sediment samples presented particular challenges related to mineral matrix interference: silicate, carbonate, and iron oxyhydroxide mineral phases produce Raman bands overlapping the 900–1200 cm⁻¹ fingerprint region of several common polymers, and density separation protocols employing NaI or ZnCl₂ solutions are essential pre-treatment steps that nonetheless introduce the risk of particle losses for high-density polymers such as PVC and PET. Atmospheric deposition studies using rooftop filter collectors provided among the highest real-sample accuracies reviewed (94%, CNN-Transformer; Li et al., 2026), reflecting the relatively controlled sampling geometry and lower degree of matrix interference compared to aquatic sediment extracts, although the sub-5-μm particle fraction—constituting over 60% of total particle count by the same study—remained analytically inaccessible to standard Raman instrumentation without SERS enhancement.
Drinking water and wastewater treatment plant effluent case studies revealed a critical tension between detection sensitivity and throughput: the very low particle concentrations in treated drinking water (reported as approximately 1 particle per litre in Sunil et al., 2025) necessitate large-volume filtration (up to 1000 L), generating filters with heterogeneous particle loading that challenges automated segmentation algorithms. Wastewater effluent, by contrast, contains high particle concentrations but also high organic loading, surfactant residues, and colloidal matter that degrade ATR crystal surfaces and introduce time-dependent spectral baseline instabilities. The RF classifier applied to wastewater FTIR data by Smolen et al. (2025) achieved only 83% accuracy—the lowest single-matrix figure in Table 5—primarily due to surfactant-induced carbonyl baseline offsets that the model, trained on clean spectra, could not compensate without explicit spectral preprocessing corrections.
Table 5. Real-world case studies of ML-assisted Raman and FTIR microplastic analysis across environmental matrices.
Matrix	Location / Region	Spectroscopy + ML Model	Accuracy / F1	Key Findings / Limitations	Reference
River water (filtered)	South Asia (Indus-like basin)	Raman + 1D-CNN	89% / 0.87	PE and PP dominant; biofouling on particles >50 μm degraded spectral quality; sub-10-μm fraction underrepresented	Zhu et al., 2025
Wastewater effluent	Urban treatment plant, Asia	ATR-FTIR + RF	83% / 0.81	High fibre count (polyester, nylon); surfactant residues caused baseline offsets in ATR spectra; 17% misclassification of weathered PET	Smolen et al., 2025
Marine sediment	Coastal South Asia	Raman mapping + 2D-CNN	91% / 0.90	Strong performance on >20 μm particles; particles <10 μm required extended laser exposure; salt crystal interference mitigated by density separation	Zhu et al., 2025
Freshwater lake sediment	Rawal Lake, Pakistan	Portable FTIR + SVM	80% / 0.78	Dominant polymer: PP from agricultural films; SVM misclassified co-polymers; field instrument noise reduced accuracy 11% vs. lab baseline	Khanam et al., 2025
Textile wastewater	Industrial discharge, Pakistan	ATR-FTIR + XGBoost	86% / 0.84	Polyester microfibers dominant (>70% by count); dye adsorption created spurious carbonyl bands; XGBoost outperformed SVM on fibre-vs-fragment discrimination	Khanam et al., 2025
Agricultural soil	Irrigated farmland, South Asia	Raman + ResNet-1D	88% / 0.86	PE mulch films and PP twine fragments dominant; humic acid fluorescence required 785 nm excitation; model retrained on weathered spectra improved accuracy 6%	Müller et al., 2025
Atmospheric deposition (filter)	Urban rooftop, East Asia	Raman mapping + CNN-Transformer	94% / 0.93	High diversity of polymer types; fibres and films distinguished by morphological features; small particle mass fraction (≤5 μm) constituted >60% by count but <5% by mass	Li et al., 2026
Marine surface water	Pacific gyre region	Raman + SVM	85% / 0.83	Extensive photooxidative weathering degraded spectral fingerprints; pristine-library SVM failed on carbonyl-bearing PE; augmented retraining partially restored performance	Zhang et al., 2025
Drinking water	Municipal supply, Europe	Raman + 1D-CNN (transfer learning)	92% / 0.91	Very low particle concentrations required large-volume filtration (1000 L); detection limit ~1 particle/L; high precision but low recall on <5 μm fraction	Sunil et al., 2025
Mixed biota (fish gut)	Coastal fisheries, South Asia	FTIR micro-ATR + RF + CNN ensemble	82% / 0.80	Biological tissue matrix co-extracted; enzymatic digestion incomplete on chitin; misclassification of nylon as biological fibre in 12% of cases	Cabot et al., 2024
Accuracy values reflect performance on field-collected samples unless otherwise noted. * Open-set accuracy including correct rejection of unknown polymer types.

4.3 South Asia and Pakistan-Specific Applications
Pakistan and the broader South Asian region constitute a high-priority context for microplastic monitoring owing to the confluence of several aggravating factors: a rapidly expanding textile and plastics manufacturing sector with limited effluent treatment infrastructure, extensive single-use plastic consumption, high riverine plastic transport loads through the Indus Basin—one of the world's most heavily polluted major river systems—and a pronounced deficit of systematic, geographically distributed monitoring data relative to the scale of contamination (Khanam et al., 2025). The Indus River and its tributaries traverse densely populated agricultural and industrial regions before emptying into the Arabian Sea, and microplastic contamination at each successive compartment—textile wastewater discharges, irrigation return flows, floodplain sediments, riparian soils, and coastal marine sediments—reflects distinct polymer fingerprints that demand spatially resolved monitoring programs.
Textile wastewater from dyeing and finishing operations represents perhaps the most analytically tractable entry point for ML-assisted spectroscopy in Pakistan: polyester (PET) and nylon microfibers released during fabric processing are present at high concentrations in effluent streams, and their size range (typically 200–2000 μm in length) is well within the detection range of ATR-FTIR equipped with accessible substrates. Khanam et al. (2025) applied an XGBoost classifier to ATR-FTIR spectra of textile effluent particles from industrial discharge sites in Pakistan, achieving 86% accuracy across six polymer classes. Critically, the study identified that reactive dye adsorption onto polyester fibre surfaces introduced spurious carbonyl bands at ~1710 cm⁻¹ that the baseline XGBoost model systematically misclassified as PET-to-PA-6 boundary cases; incorporation of a dye-adsorption spectral correction factor as an explicit input feature improved F1-score by 0.04 on the validation set. This domain-specific preprocessing insight—chemically informed feature engineering driven by knowledge of the local industrial matrix—exemplifies the kind of context-sensitive adaptation that distinguishes rigorous environmental deployment from naive laboratory model transfer.
Rawal Lake, the primary reservoir for drinking water supply to Rawalpindi and Islamabad, has been identified as a microplastic sink reflecting a diverse urban and peri-urban input mosaic: municipal solid waste littering, stormwater runoff carrying fragmented mulch films and single-use packaging, atmospheric deposition, and direct discharge from informal settlements. Khanam et al. (2025) deployed a portable FTIR instrument coupled to an SVM classifier in a field survey of Rawal Lake sediment, reporting 80% polymer-type accuracy—the lowest figure in Table 5—with the accuracy deficit attributable primarily to instrument-related spectral noise (signal-to-noise ratio approximately three-fold lower than the benchtop reference), misclassification of co-polymers absent from the SVM training library, and the high organic carbon content of lake sediment generating baseline offsets that portable ATR instruments cannot correct in real-time. Despite this accuracy limitation, the study demonstrated that portable ML-FTIR systems could reliably distinguish the dominant polymers (PP agricultural films, PE packaging fragments, polyester fibres) at an accuracy sufficient for broad-category spatial mapping—a practically meaningful outcome for prioritizing remediation interventions even in the absence of laboratory-grade precision.
The regulatory context in Pakistan compounds the analytical challenges: there are currently no legally binding microplastic concentration thresholds in drinking water, irrigation water, or sediment, and monitoring programs are largely confined to academic research initiatives without systematic government enforcement capacity (Khanam et al., 2025; Cabot et al., 2024). The deployment of portable, low-cost ML-enhanced spectrometers—operating at price points of $500–2,000 for FTIR and $2,000–5,000 for Raman—directly addresses the instrument accessibility gap in such regulatory environments, enabling distributed sampling campaigns by non-specialist field operators whose data can be centrally processed through cloud-based classification pipelines. The standardization of portable FTIR spectral formats and the development of Pakistan-specific spectral reference libraries incorporating locally dominant polymer formulations, agricultural film chemistries, and textile dye signatures represent the most critical near-term infrastructure investments for advancing microplastic monitoring capacity in the region.
4.4 Limitations and Sources of Error in Real-World Deployment
A rigorous accounting of error sources in real-world ML-spectroscopy deployments reveals a hierarchical cascade of uncertainties that compound from sample collection through spectral acquisition to model prediction. At the sampling stage, contamination with atmospheric plastic fibres during field collection, preservation, and laboratory processing introduces false-positive particles whose spectral signatures are indistinguishable from sample-derived particles; procedural blank correction protocols are inconsistently applied across the literature, and Müller et al. (2025) estimated that uncorrected blank contamination could account for up to 20% of fibre counts in low-concentration atmospheric samples. Pre-treatment steps—enzymatic digestion, density separation, chemical oxidation—differentially affect polymers of varying density, chemical stability, and surface polarity, introducing size-dependent and polymer-dependent recovery biases that are rarely quantified with sufficient rigor to permit mass-balance closure.
At the spectral acquisition stage, instrument-to-instrument variability in wavenumber calibration (±1–3 cm⁻¹), laser power stability, detector sensitivity drift, and optical alignment introduces systematic spectral offsets that shift band positions and alter relative intensities relative to the reference spectra used for model training. Portable spectrometers exhibit larger instrumental variability than benchtop systems, and models trained on high-resolution benchtop data applied to portable field spectra without explicit domain adaptation suffer accuracy penalties of 8–15 percentage points as consistently documented across studies (Khanam et al., 2025; Smolen et al., 2025). Particle orientation effects in Raman spectroscopy—arising from the polarization dependence of Raman scattering intensities for oriented crystalline or semi-crystalline polymer domains—introduce intensity variability for elongated fibres that is not captured by isotropic reference spectra in standard libraries (Sunil et al., 2025).
At the model inference stage, the closed-set assumption—whereby classifiers assign every query spectrum to one of the polymer classes present in the training library—is structurally inappropriate for environmental samples containing unknown polymer blends, additives, and degradation products. Out-of-distribution inputs are silently misclassified rather than flagged as unknowns, generating confident but incorrect predictions that can materially bias abundance estimates. The absence of calibrated uncertainty quantification (prediction confidence intervals, posterior probability distributions) in the majority of reviewed ML classifiers means that end-users receive deterministic polymer-type assignments without any indication of model confidence, a practice incompatible with the rigorous uncertainty propagation expected in regulatory-grade analytical methods (Zhang et al., 2025; Li et al., 2026). Small particle bias—the systematic underrepresentation of sub-10-μm particles in both Raman (due to fluorescence interference and long acquisition times) and FTIR (due to diffraction-limited spatial resolution) analyses—means that particle count distributions reported in the literature reflect a size-filtered view of true microplastic populations, with unknown implications for toxicological risk assessment frameworks that are increasingly focused on nanoplastic fractions.
4.5 Quantitative vs. Qualitative Detection: Progress Toward Mass Quantification
The preponderance of ML-spectroscopy studies in the microplastics field report qualitative or semi-quantitative outputs: polymer-type classifications and particle counts per unit volume, area, or mass of sample. True mass-based quantification—reporting polymer-specific mass concentrations (e.g., μg PE per litre of water, mg PP per kg sediment) rather than particle counts—is analytically more demanding but environmentally more interpretable, as it connects directly to mass-balance models of plastic transport, risk-weighted exposure assessments, and regulatory threshold frameworks that are increasingly expressed in mass concentration units (Müller et al., 2025; Li et al., 2026).
Table 6 summarizes quantification performance metrics from the reviewed literature. The most rigorous mass quantification approach identified is TGA-FTIR, in which the mass fraction of each polymer component is determined from the mass loss profile and evolved gas composition during controlled thermal decomposition; Müller et al. (2025) reported a limit of detection of 0.5 wt% for individual polymer components in mixed sediment samples, with mass fraction accuracy of ±3% on spiked validation samples—substantially superior to any spectroscopic counting approach. However, as noted in Section 3.5, TGA-FTIR is inherently destructive, bulk-averaged, and low-throughput, limiting its utility to composite sample characterization rather than single-particle analysis.
For spectroscopy-based counting approaches, conversion from particle count to mass requires assumptions about particle density, geometry (sphere, cylinder, film), and size—assumptions that introduce compounding uncertainties of typically 50–300% in mass estimates depending on particle morphology heterogeneity (Zhu et al., 2025). The CNN-Transformer atmospheric study of Li et al. (2026) estimated mass concentrations from particle volume (derived from projected area and assumed aspect ratio) multiplied by literature polymer densities, achieving reported mass concentration estimates with an uncertainty of approximately ±40%—an improvement over naive counting but still insufficient for regulatory compliance monitoring that typically demands ±20% or better. Spike-recovery experiments, in which known masses of polymer particles are added to environmental matrix blanks and recovered through the full analytical workflow, are the most robust validation approach for quantitative claims; Sunil et al. (2025) reported 78–95% spike recovery (size- and polymer-dependent) for 1D-CNN Raman analysis of drinking water, highlighting the strong size dependence of analytical recovery and the inadequacy of single-point accuracy figures as a characterization of quantitative method performance.
Table 6. Quantitative analytical performance of ML-spectroscopy methods for microplastic analysis in environmental matrices.
Model	Matrix	Polymer Classes	LOD / Quantification Metric	Accuracy on Real Samples (%)	Reference
1D-CNN (Raman)	River water	12 types	LOD ~1 particle/L (>10 μm); no mass quantification reported	89	Zhu et al., 2025
RF (ATR-FTIR)	Wastewater effluent	8 types	LOD ~5 particles/L; particle count per filter area (particles/cm²)	83	Smolen et al., 2025
2D-CNN (Raman hyperspectral)	Marine sediment	10 types	Areal density (particles/g dry weight); LOD ~0.5 μm² particle area per pixel	91	Zhu et al., 2025
XGBoost (FTIR)	Textile wastewater	6 types	Mass fraction (%) via TGA calibration curve; LOD ~0.01 mg/L	86	Khanam et al., 2025
ResNet-1D (Raman)	Agricultural soil	9 types	Particle count/kg soil; no mass LOD; estimated size bias >20 μm	88	Müller et al., 2025
CNN-Transformer (multimodal)	Atmospheric filter	15 types	Particle flux (particles/m²/day); mass concentration estimated from volume × density; LOD ~1 μg/m³	94	Li et al., 2026
TGA-FTIR ensemble (RF+CNN)	Mixed polymer sediment	7 types	Mass% per polymer type; LOD 0.5 wt%; most accurate mass quantification reported	95	Müller et al., 2025
Siamese Network (Raman)	Marine surface water	Open-set (14 known + unknowns)	Presence/absence per class; no mass metric; LOD ~2 particles/L	85*	Zhang et al., 2025
1D-CNN transfer (Raman)	Drinking water	10 types	LOD ~1 particle/L; mass quantification not reported; detection validated by spike-recovery at 5 particles/L	92	Sunil et al., 2025
LOD = limit of detection. * Open-set accuracy including correct rejection of unknown polymer types. Mass-based LODs are reported where available; particle-count-based LODs are reported otherwise.

4.6 Transition to Future Directions
The body of benchmarking evidence assembled in this section permits several high-confidence conclusions and identifies a clear set of priorities for methodological advancement. It is unambiguous that ML-assisted vibrational spectroscopy has transformed microplastic analysis from a low-throughput, expert-dependent procedure into an automated, scalable analytical pipeline capable of processing thousands of particles per day with polymer-type accuracies exceeding 90% on field samples—an advance of fundamental importance for the generation of the large, spatially resolved datasets that environmental risk models and regulatory frameworks require. The demonstrated feasibility of portable FTIR instruments coupled to cloud-based ML classifiers for field campaigns in South Asia underscores the potential for globally equitable monitoring capacity that is not confined to well-resourced laboratory environments (Khanam et al., 2025; Cabot et al., 2024).
Nonetheless, the persistent accuracy gap between laboratory and environmental conditions, the absence of standardized open benchmark datasets, the systematic underestimation of sub-10-μm particles, the lack of calibrated uncertainty quantification, and the inadequacy of current mass quantification protocols collectively define a research agenda that is both technically demanding and urgently consequential. Progress on these fronts requires coordinated investments in federated spectral database infrastructure, standardized weathering protocols for model training augmentation, domain-adaptation frameworks for portable instrument deployment, and regulatory engagement to define fitness-for-purpose accuracy thresholds for specific monitoring applications. The integration of XAI tools to provide chemically interpretable, auditable model outputs is a prerequisite for the transition of ML-spectroscopy methods from research demonstrations to accredited analytical standards.
The benchmarking and case studies presented here highlight both the transformative potential and persistent challenges of ML-enhanced spectroscopy, paving the way for future directions in portable, sustainable, and globally equitable microplastic monitoring.
Disclaimer
All statements in this section are based exclusively on the cited references.
